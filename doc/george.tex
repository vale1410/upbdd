\documentclass{article}



\begin{document}

\section {SAT-encoding of ROBDD}

The SAT-encoding is based on this idea, that every path from the root of the 
ROBDD to the 1-sink represents a model (or a class of models, if we admit 
shortcuts/longedges). 

For every BDD node $N$ we introduce a Boolean variable $c_N$ with the meaning: 

\begin{equation}
  \neg c_N \Longleftrightarrow \mbox{no path from root to 1-sink can go throuh $N$}
\end{equation}


We distinguish between reachability upwards (from the 1-sink) and downwards 
(from the root). First we consider reachability from the 1-sink and assume $N$ 
is a node with decision variable $x$ and successor nodes $T$ (for 
\textit{high-successor}) and $F$ (for \textit{low-successor}). Obviously, the 
0-node is not reachable from the 1-sink, that means $\neg N_0$ is a unit 
clause, and the 1-sink is reachable. We obtain the following rules for the 
inner nodes. 

\begin{itemize}
\item if both successors are not reachable, then the node itself is also not reachable:
\begin{equation}
  \neg c_T\wedge\neg c_F\Rightarrow \neg c_N
\end{equation}
\item if the the high-successor is not reachable and the decision variable is known to be true, then $N$ is not reachable:
\begin{equation}
  \neg c_T\wedge x\Rightarrow \neg c_N
\end{equation}
\item analogously for the low-successor:
\begin{equation}
  \neg c_F\wedge \neg x\Rightarrow \neg c_N
\end{equation}
\end{itemize}

Now we consider reachability from the root. Obviously the root $R$ itself is 
reachable, thus $c_R$ is a unit clause. For all other nodes we observe the 
following condition. We assume $P_1, \ldots, P_k$ are the parents of $N$ and 
$l_1,\ldots, l_k$ are the decisions that are necessary to go from $P_i$ to $N$. 
More precisely, if $l_i$ is a positive literal, then $N$ is the high-successor 
of $P_i$ (case for $l_i$ negative is analogous). Then $N$ is not reachable 
passing $P_i$ if $P_i$ is not reachable or $l_i$ is known to be false: 
\begin{equation}
  \bigwedge_{i=1}^k{(\neg c_{P_i}\vee\neg l_i)}\Rightarrow \neg c_N
\end{equation}

Of course this rule has to be expanded by a Tseitsin transformation to CNF. 

Now we need one more rule to prune variables also in from the BDD's domain. 
Basically we can prune a variable $x$ if it is known that no path from the root 
to the 1-sink can go through an edge that is labelled with $x$ (analogously for 
$\neg x$). We have to pay special attention to longedges that jump the variable 
$x$ in the BDD. Let $(A_1, l_1, B_1),\ldots,(A_k, l_k, B,_k)$ be the set of 
longedges that jump $x$, in particular in node $A_i$ the high-successor (if 
$l_i$ is positive) is $B_i$. Further let $(C_1, x, D_1),\ldots, (C_m, x, D_m)$ 
be the set of all edges that are labeled with $x$. We have now: 

\begin{equation}
\bigwedge_{i=1}^k{(\neg c_{A_i}\vee\neg c_{B_i}\vee\neg l_i)}\wedge\bigwedge_{i=1}^m{(\neg c_{A_i}\vee\neg c_{B_i})}\Rightarrow \neg x
\end{equation}







\end{document}



